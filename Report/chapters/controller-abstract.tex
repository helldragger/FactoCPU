\chapter{Résumé}

\paragraph{Une machine de Turing complète?}

Afin de pouvoir éxecuter n'importe quel langage sur notre machine, nous devons créer une machine capable d'émuler n'importe quel langage. Autrement dit, de pouvoir lire le fonctionnement d'une machine et de l'interprèter afin de reproduire les mêmes résultats que si on l avait construite indépendamment.

Une machine de Turing pouvant alors émuler toute autre machine de Turing est une machine de Turing dite complète.
Et en créer une à partir de Factorio est l'objectif de ce projet.

\section{En quelques nombres}
% temps par micro instruction * micro instructions par cycle => nombre de cycles par secondes (Hz)
% taille d'une cellule memoire * nb adresses en 32 bit => taille totale potentielle en memoire

\section{Survol de l'organisation spatiale}

\section{Résumé des composants}
	
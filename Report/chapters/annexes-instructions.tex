\chapter{Instructions}
\paragraph{This is what programs can use}
Those instructions are the lowest level instructions programs can use to write their logic with. 
They are defined in the microcode using sequences of micro- instructions, which can be seen in the next annexe.
Those instructions are sole the content of a single compiled program memory cell.

\textbf{$A$} refers to the content of the register A.

\textbf{$PC$} refers to the content of the register PC.

\textbf{$V$} refers to the content of the register V.

\textbf{$V_{mask}$} refers to the content of the register MASK.

\textbf{$MEM[1]$} refers to the memory cell content at the address contained in signal [1].

The symbol $\vec{ }$ means the value will be considered as a vector.

\paragraph{Unconditional jump}
As an experimentation, a unconditional jump can be forced by using the signal $[J]$, containing the next instruction address. 

\section{Memory}
\begin{tabular}{l | l | l  l}
[E] & NAME & ARGS [signal]:type & DESC \\
01 & LOAD-A & [1]:address & $A = MEM[1]$\\
02 & STORE-A & [1]:address & $MEM[1] = A$\\
18 & LOAD-VEC & [1]:address & $\vec{V} = \vec{MEM[1]}$\\
19 & LOAD-VMASK & [1]:address & $\vec{V_{mask}} = \vec{MEM[1]}$\\
20 & STORE-VEC & [1]:address & $\vec{MEM[1]} = \vec{V}$\\
21 & STORE-VMASK & [1]:address & $\vec{MEM[1]} = \vec{V_{mask}}$\\
\end{tabular}

\section{Jump and conditionals}
\begin{tabular}{l | l | l  l}
[E] & NAME & ARGS [signal]:type & DESC \\
-- & JMP & [J]:address & PC = [J]\\
03 & JMP-A-EQ & [1]:address, [2]:address & IF ($A = MEM[2]$): PC = [1]\\
04 & JMP-A-LT & [1]:address, [2]:address & IF ($A < MEM[2]$): PC = [1]\\
05 & JMP-A-LE & [1]:address, [2]:address & IF ($A \le MEM[2]$): PC = [1]\\
06 & JMP-A-GT & [1]:address, [2]:address & IF ($A > MEM[2]$): PC = [1]\\
07 & JMP-A-GE & [1]:address, [2]:address & IF ($A \ge MEM[2]$): PC = [1]\\
08 & JMP-A-NE & [1]:address, [2]:address & IF ($A \ne MEM[2]$): PC = [1]\\
10 & JMP-V-EQ & [1]:address & IF ($A = MEM[1]$): PC = A\\
11 & JMP-V-LT & [1]:address & IF ($A < MEM[1]$): PC = A\\
12 & JMP-V-LE & [1]:address & IF ($A \le MEM[1]$): PC = A\\
13 & JMP-V-GT & [1]:address & IF ($A > MEM[1]$): PC = A\\
14 & JMP-V-GE & [1]:address & IF ($A \ge MEM[1]$): PC = A\\
15 & JMP-V-NE & [1]:address & IF ($A \ne MEM[1]$): PC = A\\
\end{tabular}

\section{Arithmetic}
\begin{tabular}{l | l | l  l}
[E] & NAME & ARGS [signal]:type & DESC \\
09 & ARITHM & [1]:address, [2]:value=1 & $A = A \mathbf{*} MEM[1]$\\  
09 & ARITHM & [1]:address, [2]:value=2 & $A = A \mathbf{/} MEM[1]$\\  
09 & ARITHM & [1]:address, [2]:value=3 & $A = A \mathbf{+} MEM[1]$\\  
09 & ARITHM & [1]:address, [2]:value=4 & $A = A \mathbf{-} MEM[1]$\\  
09 & ARITHM & [1]:address, [2]:value=5 & $A = A \mathbf{\%} MEM[1]$\\  
09 & ARITHM & [1]:address, [2]:value=6 & $A = A^{MEM[1]}$\\  
09 & ARITHM & [1]:address, [2]:value=7 & $A = A \mathbf{<<} MEM[1]$\\  
09 & ARITHM & [1]:address, [2]:value=8 & $A = A \mathbf{>>} MEM[1]$\\  
09 & ARITHM & [1]:address, [2]:value=9 & $A = A \mathbf{\&} MEM[1]$\\  
09 & ARITHM & [1]:address, [2]:value=10 & $A = A \mathbf{|} MEM[1]$\\  
09 & ARITHM & [1]:address, [2]:value=11 & $A = A \mathbf{xor} MEM[1]$\\  
16 & NOT-A & & $A = !A$\\ 
\end{tabular}

\section{Vectorial}
\begin{tabular}{l | l | l  l}
[E] & NAME & ARGS [signal]:type & DESC \\
17 & VEC-VEC & [1]:address, [2]:value=1 & $\vec{V} = (\vec{V}\&\vec{V_{mask}}) \mathbf{*} \vec{MEM[1]}$\\  
17 & VEC-VEC & [1]:address, [2]:value=2 & $\vec{V} = (\vec{V}\&\vec{V_{mask}}) \mathbf{/} \vec{MEM[1]}$\\  
17 & VEC-VEC & [1]:address, [2]:value=3 & $\vec{V} = (\vec{V}\&\vec{V_{mask}}) \mathbf{+} \vec{MEM[1]}$\\  
17 & VEC-VEC & [1]:address, [2]:value=4 & $\vec{V} = (\vec{V}\&\vec{V_{mask}}) \mathbf{-} \vec{MEM[1]}$\\  
17 & VEC-VEC & [1]:address, [2]:value=5 & $\vec{V} = (\vec{V}\&\vec{V_{mask}}) \mathbf{\%} \vec{MEM[1]}$\\  
17 & VEC-VEC & [1]:address, [2]:value=6 & $\vec{V} = (\vec{V}\&\vec{V_{mask}})^{\vec{MEM[1]}}$\\  
17 & VEC-VEC & [1]:address, [2]:value=7 & $\vec{V} = (\vec{V}\&\vec{V_{mask}}) \mathbf{<<}\vec{MEM[1]}$\\  
17 & VEC-VEC & [1]:address, [2]:value=8 & $\vec{V} = (\vec{V}\&\vec{V_{mask}}) \mathbf{>>}\vec{MEM[1]}$\\  
17 & VEC-VEC & [1]:address, [2]:value=9 & $\vec{V} = (\vec{V}\&\vec{V_{mask}}) \mathbf{\&}\vec{MEM[1]}$\\  
17 & VEC-VEC & [1]:address, [2]:value=10 &$\vec{V} = (\vec{V}\&\vec{V_{mask}}) \mathbf{|}\vec{MEM[1]}$\\  
17 & VEC-VEC & [1]:address, [2]:value=11 & $\vec{V} = (\vec{V}\&\vec{V_{mask}})  \mathbf{xor}\vec{MEM[1]}$\\  
22 & NOT-VEC & & $\vec{V} = !(\vec{V}\&\vec{V_{mask}})$\\ 
\end{tabular}